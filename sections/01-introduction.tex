\chapter{INTRODUÇÃO}

O mercado de assessoria financeira no Brasil apresenta uma expansão
significativa, com um crescimento de 7\% no número de profissionais
certificados em 2024, alcançando 26.681 assessores, frente aos 24.931
registrados em 2023, conforme dados da ANCORD (Associação Nacional das
Corretoras e Distribuidoras de Títulos e Valores Mobiliários, Câmbio e
Mercadorias). Esse aumento evidencia a demanda crescente por serviços
financeiros personalizados e destaca a necessidade de compreender os
fatores que impulsionam o sucesso dos assessores. Segundo \cite{foerster2017}, a variação no desempenho desses profissionais está mais
associada às suas próprias características --- como habilidades,
experiência e estratégias de atendimento --- do que às dos clientes, o
que abre espaço para o uso de técnicas de machine learning, como LASSO e
Regressão Quantílica, na identificação de perfis de alta performance.
Essa abordagem pode otimizar o atendimento e aumentar a captação de
recursos, aproveitando o potencial de dados internos e de mercado.

\section{Objetivo Geral}

Descrever o processo de construção de um modelo para identificar perfis
de alta performance em assessoria financeira. Este processo será baseado
em técnicas de machine learning, utilizando dados de CRM, informações da
corretora e métricas internas de operações, com o propósito de aprimorar
estratégias de atendimento e maximizar a captação de recursos.

\section{Objetivos Específicos}

\begin{enumerate}
  \item Determinar os principais fatores que contribuem para o desempenho
  superior na captação de recursos, aplicando técnicas de seleção de
  variáveis, como LASSO, e análises estatísticas a dados históricos
  de CRM, informações institucionais de corretora (ativos de
  clientes e receitas geradas) e métricas operacionais internas
  (características demográficas dos assessores e metas definidas).

  \item Classificar os perfis de assessores e clientes de alta performance
  por meio do algoritmo de clustering em Regressão Quantílica,
  criando taxonomias que facilitem o pareamento otimizado entre
  cliente e assessor no processo de atendimento.
\end{enumerate}

% Adicione informações sobre as fontes de dados que você tem disponíveis
\section{Fontes de Dados}

Para realização desta pesquisa, serão utilizados dados do período de 2022/1 a 2026/2, com granularidade diária para todas as variáveis, provenientes das seguintes fontes:

\subsection{CRM Salesforce Cloud}
\begin{itemize}
  \item \textbf{Ligações}: origem, destino e conteúdo
  \item \textbf{Reuniões}: origem, destino, conteúdo e demais dados relevantes disponíveis
\end{itemize}

\subsection{Corretora/Banking XP}
\begin{itemize}
  \item \textbf{Ativos}: custódia e detalhes do ativo
  \item \textbf{Movimentação de Recursos}: entre instituições e detalhes da movimentação (tipo, volume, descrição)
  \item \textbf{Receitas}: todas as informações descritas com granularidade diária, segmentadas por cliente
\end{itemize}

\subsection{Operacionais}
\begin{itemize}
  \item \textbf{Informações Demográficas dos Usuários}: gênero, idade e início do trabalho na função
\end{itemize}