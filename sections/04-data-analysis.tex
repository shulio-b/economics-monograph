\chapter{ANÁLISE DE DADOS}

Este capítulo apresentará os resultados da aplicação das técnicas de machine learning para a identificação de perfis de alta performance na assessoria financeira. Serão analisados os fatores determinantes do desempenho e os perfis identificados.

\section{Análise Descritiva dos Dados}

[Esta seção será desenvolvida após a coleta e análise preliminar dos dados, incluindo estatísticas descritivas, visualizações e identificação de padrões iniciais.]

\section{Resultados da Seleção de Variáveis com LASSO}

[Esta seção apresentará os resultados da aplicação do LASSO para identificar as variáveis mais relevantes para o desempenho dos assessores. Serão discutidos os coeficientes das variáveis selecionadas e sua importância relativa.]

\section{Perfis Identificados pela Regressão Quantílica}

[Esta seção descreverá os diferentes perfis de assessores e clientes identificados pela Regressão Quantílica, destacando as características distintivas de cada grupo e suas implicações para a estratégia de atendimento.]

\section{Análise de Desempenho por Perfil}

[Esta seção analisará o desempenho de cada perfil identificado, comparando métricas de captação de recursos, retenção de clientes e outras medidas relevantes.]