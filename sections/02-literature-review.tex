\chapter{REVISÃO DA LITERATURA}

\section{Assessoria Financeira e o Papel do Assessor}

A literatura sobre assessoria financeira destaca a importância do assessor no processo de orientação financeira dos clientes. \cite{foerster2017} demonstram que os assessores exercem influência substancial sobre a alocação de ativos de seus clientes, mas fornecem customização limitada. Os efeitos fixos do assessor explicam consideravelmente mais variação no risco de portfólio e viés doméstico do que um amplo conjunto de atributos do investidor, incluindo tolerância ao risco, idade, horizonte de investimento e sofisticação financeira.

\section{Machine Learning na Previsão de Desempenho}

As técnicas de machine learning têm sido cada vez mais aplicadas na previsão de desempenho profissional. \cite{lather2020} exploraram a predição de desempenho de funcionários utilizando técnicas de machine learning, demonstrando a eficácia dessas abordagens para identificar padrões e fatores determinantes do sucesso profissional.

Mais recentemente, \cite{uppal2024} investigaram abordagens baseadas em machine learning para aprimorar a gestão de recursos humanos, utilizando sistemas automatizados de previsão de desempenho dos funcionários. Estas pesquisas oferecem insights valiosos sobre como aplicar técnicas semelhantes no contexto da assessoria financeira.

\section{Perfilamento de Clientes no Setor Financeiro}

O perfilamento de clientes no setor financeiro tem avançado significativamente com o uso de técnicas de análise exploratória e clustering. \cite{choi2024} apresentaram uma análise exploratória de dados baseada em redes e um clustering profundo de três estágios para perfilamento de clientes financeiros, oferecendo insights metodológicos que podem ser adaptados para a identificação de perfis tanto de clientes quanto de assessores.

\section{Estratégias de Atendimento ao Cliente no Mercado Financeiro}

As estratégias de atendimento ao cliente no mercado financeiro foram estudadas por \cite{knapp2007}, que desenvolveu o método Supernova Advisor, aplicado no banco Merrill Lynch na década de 80. Este método propõe padrões de atendimento para profissionais do mercado financeiro e segmentação de clientes, visando um serviço excepcional e crescimento consistente.