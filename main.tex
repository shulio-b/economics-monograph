\documentclass[12pt, openright, oneside, a4paper, chapter=TITLE, english, brazil]{abntex2}

% Inclui o preâmbulo com pacotes e configurações
\include{preambulo}

\begin{document}

% Elementos pré-textuais
\imprimircapa
\imprimirfolhaderosto

% Resumo em português
\begin{resumo}
O mercado de assessoria financeira no Brasil apresenta uma expansão
significativa, com um crescimento de 7\% no número de profissionais
certificados em 2024, alcançando 26.681 assessores, frente aos 24.931
registrados em 2023, conforme dados da ANCORD. Este trabalho busca descrever 
o processo de construção de um modelo para identificar perfis de alta 
performance em assessoria financeira, utilizando técnicas de machine learning, 
como LASSO e Regressão Quantílica. O objetivo é criar taxonomias que facilitem 
o pareamento otimizado entre cliente e assessor no processo de atendimento, 
assim como aprimorar estratégias para maximizar a captação de recursos.

\vspace{\onelineskip}
\noindent
\textbf{Palavras-chave}: Assessoria financeira. Machine learning. LASSO. Regressão Quantílica. Perfis de alta performance.
\end{resumo}

% Sumário
\pdfbookmark[0]{\contentsname}{toc}
\tableofcontents
\cleardoublepage

% Elementos textuais
\textual

\chapter{INTRODUÇÃO}

O mercado de assessoria financeira no Brasil apresenta uma expansão
significativa, com um crescimento de 7\% no número de profissionais
certificados em 2024, alcançando 26.681 assessores, frente aos 24.931
registrados em 2023, conforme dados da ANCORD (Associação Nacional das
Corretoras e Distribuidoras de Títulos e Valores Mobiliários, Câmbio e
Mercadorias). Esse aumento evidencia a demanda crescente por serviços
financeiros personalizados e destaca a necessidade de compreender os
fatores que impulsionam o sucesso dos assessores. Segundo \cite{foerster2017}, a variação no desempenho desses profissionais está mais
associada às suas próprias características --- como habilidades,
experiência e estratégias de atendimento --- do que às dos clientes, o
que abre espaço para o uso de técnicas de machine learning, como LASSO e
Regressão Quantílica, na identificação de perfis de alta performance.
Essa abordagem pode otimizar o atendimento e aumentar a captação de
recursos, aproveitando o potencial de dados internos e de mercado.

\section{Objetivo Geral}

Descrever o processo de construção de um modelo para identificar perfis
de alta performance em assessoria financeira. Este processo será baseado
em técnicas de machine learning, utilizando dados de CRM, informações da
corretora e métricas internas de operações, com o propósito de aprimorar
estratégias de atendimento e maximizar a captação de recursos.

\section{Objetivos Específicos}

\begin{enumerate}
  \item Determinar os principais fatores que contribuem para o desempenho
  superior na captação de recursos, aplicando técnicas de seleção de
  variáveis, como LASSO, e análises estatísticas a dados históricos
  de CRM, informações institucionais de corretora (ativos de
  clientes e receitas geradas) e métricas operacionais internas
  (características demográficas dos assessores e metas definidas).

  \item Classificar os perfis de assessores e clientes de alta performance
  por meio do algoritmo de clustering em Regressão Quantílica,
  criando taxonomias que facilitem o pareamento otimizado entre
  cliente e assessor no processo de atendimento.
\end{enumerate}

% Adicione informações sobre as fontes de dados que você tem disponíveis
\section{Fontes de Dados}

Para realização desta pesquisa, serão utilizados dados do período de 2022/1 a 2026/2, com granularidade diária para todas as variáveis, provenientes das seguintes fontes:

\subsection{CRM Salesforce Cloud}
\begin{itemize}
  \item \textbf{Ligações}: origem, destino e conteúdo
  \item \textbf{Reuniões}: origem, destino, conteúdo e demais dados relevantes disponíveis
\end{itemize}

\subsection{Corretora/Banking XP}
\begin{itemize}
  \item \textbf{Ativos}: custódia e detalhes do ativo
  \item \textbf{Movimentação de Recursos}: entre instituições e detalhes da movimentação (tipo, volume, descrição)
  \item \textbf{Receitas}: todas as informações descritas com granularidade diária, segmentadas por cliente
\end{itemize}

\subsection{Operacionais}
\begin{itemize}
  \item \textbf{Informações Demográficas dos Usuários}: gênero, idade e início do trabalho na função
\end{itemize}
\chapter{REVISÃO DA LITERATURA}

\section{Assessoria Financeira e o Papel do Assessor}

A literatura sobre assessoria financeira destaca a importância do assessor no processo de orientação financeira dos clientes. \cite{foerster2017} demonstram que os assessores exercem influência substancial sobre a alocação de ativos de seus clientes, mas fornecem customização limitada. Os efeitos fixos do assessor explicam consideravelmente mais variação no risco de portfólio e viés doméstico do que um amplo conjunto de atributos do investidor, incluindo tolerância ao risco, idade, horizonte de investimento e sofisticação financeira.

\section{Machine Learning na Previsão de Desempenho}

As técnicas de machine learning têm sido cada vez mais aplicadas na previsão de desempenho profissional. \cite{lather2020} exploraram a predição de desempenho de funcionários utilizando técnicas de machine learning, demonstrando a eficácia dessas abordagens para identificar padrões e fatores determinantes do sucesso profissional.

Mais recentemente, \cite{uppal2024} investigaram abordagens baseadas em machine learning para aprimorar a gestão de recursos humanos, utilizando sistemas automatizados de previsão de desempenho dos funcionários. Estas pesquisas oferecem insights valiosos sobre como aplicar técnicas semelhantes no contexto da assessoria financeira.

\section{Perfilamento de Clientes no Setor Financeiro}

O perfilamento de clientes no setor financeiro tem avançado significativamente com o uso de técnicas de análise exploratória e clustering. \cite{choi2024} apresentaram uma análise exploratória de dados baseada em redes e um clustering profundo de três estágios para perfilamento de clientes financeiros, oferecendo insights metodológicos que podem ser adaptados para a identificação de perfis tanto de clientes quanto de assessores.

\section{Estratégias de Atendimento ao Cliente no Mercado Financeiro}

As estratégias de atendimento ao cliente no mercado financeiro foram estudadas por \cite{knapp2007}, que desenvolveu o método Supernova Advisor, aplicado no banco Merrill Lynch na década de 80. Este método propõe padrões de atendimento para profissionais do mercado financeiro e segmentação de clientes, visando um serviço excepcional e crescimento consistente.
\chapter{METODOLOGIA}

Este capítulo apresenta a metodologia adotada para identificar perfis de alta performance na assessoria financeira, detalhando as técnicas de machine learning utilizadas, o processo de coleta e tratamento dos dados, e os métodos de análise.

\section{Coleta e Preparação dos Dados}

Os dados utilizados nesta pesquisa serão provenientes de três fontes principais: CRM Salesforce Cloud, Corretora/Banking XP e dados operacionais internos. O período de análise compreende de 2022/1 a 2026/2, com granularidade diária para todas as variáveis.

\section{Técnicas de Machine Learning}

\subsection{LASSO (Least Absolute Shrinkage and Selection Operator)}

O LASSO será utilizado para a seleção de variáveis relevantes, permitindo identificar quais fatores são mais determinantes para o desempenho na captação de recursos. Esta técnica é particularmente útil quando se trabalha com um grande número de variáveis potencialmente correlacionadas, pois realiza automaticamente a seleção de variáveis e a regularização, ajudando a prevenir o sobreajuste do modelo.

A formulação matemática do LASSO é dada por:

\begin{equation}
\hat{\beta}^{lasso} = \arg\min_{\beta} \left\{ \sum_{i=1}^{N} (y_i - \beta_0 - \sum_{j=1}^{p} \beta_j x_{ij})^2 + \lambda \sum_{j=1}^{p} |\beta_j| \right\}
\end{equation}

onde $\lambda$ é o parâmetro de regularização que controla a força da penalidade.

\subsection{Regressão Quantílica para Clustering}

A Regressão Quantílica será empregada para identificar padrões e criar agrupamentos (clusters) de assessores e clientes com base em seus perfis e desempenho. Esta técnica permite a análise de diferentes quantis da distribuição de desempenho, fornecendo uma visão mais completa dos fatores que influenciam o sucesso em diferentes níveis de performance.

A função objetivo da Regressão Quantílica para o quantil $\tau$ é dada por:

\begin{equation}
\min_{\beta} \sum_{i:y_i \geq x_i'\beta} \tau |y_i - x_i'\beta| + \sum_{i:y_i < x_i'\beta} (1-\tau) |y_i - x_i'\beta|
\end{equation}

\section{Validação dos Modelos}

Para garantir a robustez dos resultados, serão utilizadas técnicas de validação cruzada, como k-fold cross-validation. Além disso, serão aplicados testes estatísticos para verificar a significância dos fatores identificados e a estabilidade dos agrupamentos encontrados.

\section{Métricas de Avaliação}

O desempenho dos assessores será avaliado principalmente através de métricas relacionadas à captação de recursos, considerando tanto o volume absoluto quanto o crescimento relativo. Outras métricas secundárias incluirão a retenção de clientes, a diversificação da carteira e a satisfação do cliente, quando disponíveis.
\chapter{ANÁLISE DE DADOS}

Este capítulo apresentará os resultados da aplicação das técnicas de machine learning para a identificação de perfis de alta performance na assessoria financeira. Serão analisados os fatores determinantes do desempenho e os perfis identificados.

\section{Análise Descritiva dos Dados}

[Esta seção será desenvolvida após a coleta e análise preliminar dos dados, incluindo estatísticas descritivas, visualizações e identificação de padrões iniciais.]

\section{Resultados da Seleção de Variáveis com LASSO}

[Esta seção apresentará os resultados da aplicação do LASSO para identificar as variáveis mais relevantes para o desempenho dos assessores. Serão discutidos os coeficientes das variáveis selecionadas e sua importância relativa.]

\section{Perfis Identificados pela Regressão Quantílica}

[Esta seção descreverá os diferentes perfis de assessores e clientes identificados pela Regressão Quantílica, destacando as características distintivas de cada grupo e suas implicações para a estratégia de atendimento.]

\section{Análise de Desempenho por Perfil}

[Esta seção analisará o desempenho de cada perfil identificado, comparando métricas de captação de recursos, retenção de clientes e outras medidas relevantes.]
\chapter{CONCLUSÃO}

[A conclusão será desenvolvida após a realização da análise de dados, sintetizando os principais achados da pesquisa e suas implicações para a prática da assessoria financeira.]

\section{Principais Achados}

[Esta seção resumirá os principais resultados encontrados, destacando os fatores mais relevantes para o desempenho dos assessores e os perfis de maior sucesso.]

\section{Implicações Práticas}

[Esta seção discutirá as implicações práticas dos resultados para as estratégias de atendimento e captação de recursos em assessoria financeira.]

\section{Limitações e Pesquisas Futuras}

[Esta seção abordará as limitações do estudo e sugerirá direções para pesquisas futuras nesta área.]

% Elementos pós-textuais
\postextual
\bibliography{references}

\end{document}